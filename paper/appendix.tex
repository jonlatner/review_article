\section{Appendix tables (online supplement)}
\label{appendix_tables}

{\footnotesize %
\begin{longtable}{l>{\raggedright\arraybackslash}p{1.2in}>{\raggedright\arraybackslash}p{.9in}>{\raggedright\arraybackslash}p{.9in}>{\raggedright\arraybackslash}p{1.2in}ll}
\caption{Empirical findings from the literature on the consequences of temporary employment, by outcome and split by geographic region and direction of comparison} \\ 
   
\label{table_articles_theory}
\\ \hline \\ 
 [-1.8ex]\rowcolor{white} 
\multicolumn{3}{l}{Article characteristics} 
& \multicolumn{3}{l}{Indicator characteristics} 
& \multicolumn{1}{l}{Evidence}
\\ 

            \cmidrule(lr){1-3} 
            \cmidrule(lr){4-6}
            \cmidrule(lr){7-7}


\rowcolor{white}ID 
& Article 
& Country
& Region 
& \multicolumn{1}{>{\raggedright\arraybackslash}p{1.2in}}{Indicator}
& Comparison
\\ 
\hline
\endfirsthead

 
\rowcolor{white}\multicolumn{7}{@{}l}{\ldots Table \ref{table_articles_theory} continued} \\
\hline
\rowcolor{white}
\multicolumn{3}{l}{Article characteristics} 
& \multicolumn{3}{l}{Indicator characteristics} 
& \multicolumn{1}{l}{Evidence}
\\ 

            \cmidrule(lr){1-3} 
            \cmidrule(lr){4-6}
            \cmidrule(lr){7-7}


\rowcolor{white}ID 
& Article 
& Country
& Region 
& \multicolumn{1}{>{\raggedright\arraybackslash}p{1.2in}}{Indicator}
& Comparison
\\
\hline
\endhead % all the lines above this will be repeated on every page
\hline
\rowcolor{white}\multicolumn{7}{r@{}}{Table \ref{table_articles_theory} continued \ldots}\\
\endfoot
\hline
\endlastfoot

 \hline
  1 & \citealp{amuedo_dorantes_2000} & ES & Southern &  & Upwards & Segmentation \\ 
    2 & \citealp{amuedo_dorantes_serrano_padial_2007} & ES & Southern & job movers & None & Segmentation \\ 
    2 & \citealp{amuedo_dorantes_serrano_padial_2007} & ES & Southern & job stayers & None & Integration \\ 
    3 & \citealp{arranz_etal_2010} & ES & Southern &  & None & Segmentation \\ 
    4 & \citealp{babos_2014} & 8 CEE countries & Eastern &  & Upwards & Segmentation \\ 
    5 & \citealp{baranowska_etal_2011} & PL & Continental &  & None & Segmentation \\ 
    6 & \citealp{barbieri_cutuli_2016} & 13 European countries & Southern &  & Upwards & Segmentation \\ 
    6 & \citealp{barbieri_cutuli_2016} & 13 European countries & Continental &  & Downwards & Integration \\ 
    6 & \citealp{barbieri_cutuli_2016} & 13 European countries & Southern &  & Downwards & Integration \\ 
    6 & \citealp{barbieri_cutuli_2016} & 13 European countries & Northern &  & Upwards & Segmentation \\ 
    6 & \citealp{barbieri_cutuli_2016} & 13 European countries & Northern &  & Downwards & Integration \\ 
    6 & \citealp{barbieri_cutuli_2016} & 13 European countries & Continental &  & Upwards & Segmentation \\ 
    7 & \citealp{barbieri_cutuli_2018} & IT & Southern &  & Upwards & Segmentation \\ 
    8 & \citealp{barbieri_scherer_2009} & IT & Southern &  & Downwards & Segmentation \\ 
    8 & \citealp{barbieri_scherer_2009} & IT & Southern &  & Upwards & Segmentation \\ 
    9 & \citealp{barbieri_sestito_2008} & IT & Southern &  & Downwards & Integration \\ 
   10 & \citealp{berson_2018} & FR & Continental &  & Upwards & Segmentation \\ 
   11 & \citealp{berton_etal_2011} & IT & Southern & Type of FTC: fixed-term jobs, apprenticeship, and training programmes & Upwards & Integration \\ 
   11 & \citealp{berton_etal_2011} & IT & Southern & Type of FTC: freelance contracts & Upwards & Segmentation \\ 
   12 & \citealp{booth_etal_2002} & UK & Northern &  & Upwards & Integration \\ 
   13 & \citealp{bosco_valeriani_2018} & IT & Southern &  & Upwards & Segmentation \\ 
   14 & \citealp{brown_sessions_2003} & UK & Northern &  & Upwards & Segmentation \\ 
   15 & \citealp{comi_grasseni_2012} & 9 European countries & Northern &  & Upwards & Segmentation \\ 
   15 & \citealp{comi_grasseni_2012} & 9 European countries & Southern &  & Upwards & Segmentation \\ 
   15 & \citealp{comi_grasseni_2012} & 9 European countries & Eastern &  & Upwards & Segmentation \\ 
   16 & \citealp{de_graaf_zijl_etal_2011} & NL & Continental &  & Downwards & Integration \\ 
   17 & \citealp{de_lange_etal_2014} & NL & Continental &  & Upwards & Integration \\ 
   18 & \citealp{debels_2008} & EU & Northern & men & None & Integration \\ 
   18 & \citealp{debels_2008} & EU & Southern & men & None & Segmentation \\ 
   18 & \citealp{debels_2008} & EU & Northern & women & None & null \\ 
   18 & \citealp{debels_2008} & EU & Southern & women & None & null \\ 
   19 & \citealp{gagliarducci_2005} & IT & Southern &  & Upwards & Segmentation \\ 
   20 & \citealp{gash_2008} & FR & Continental &  & None & Integration \\ 
   20 & \citealp{gash_2008} & DE-W & Continental &  & None & Integration \\ 
   20 & \citealp{gash_2008} & UK & Northern &  & None & Integration \\ 
   20 & \citealp{gash_2008} & DK & Continental &  & None & Integration \\ 
   21 & \citealp{gash_mcginnity_2007} & DE & Continental & women & Upwards & Segmentation \\ 
   21 & \citealp{gash_mcginnity_2007} & DE & Continental & men & Upwards & Segmentation \\ 
   21 & \citealp{gash_mcginnity_2007} & FR & Continental & women & Upwards & Segmentation \\ 
   21 & \citealp{gash_mcginnity_2007} & FR & Continental & men & Upwards & Segmentation \\ 
   22 & \citealp{gebel_2009} & DE-W & Continental &  & Upwards & Segmentation \\ 
   23 & \citealp{gebel_2010} & UK & Northern &  & Upwards & Integration \\ 
   23 & \citealp{gebel_2010} & DE & Continental &  & Upwards & Segmentation \\ 
   24 & \citealp{gebel_2013} & DE-W & Continental &  & Downwards & Integration \\ 
   24 & \citealp{gebel_2013} & UK & Northern &  & Downwards & Integration \\ 
   24 & \citealp{gebel_2013} & CH & Continental &  & Downwards & null \\ 
   25 & \citealp{giesecke_gross_2003} & DE & Continental &  & Upwards & Segmentation \\ 
   26 & \citealp{giesecke_gross_2004} & DE & Continental &  & Upwards & Segmentation \\ 
   26 & \citealp{giesecke_gross_2004} & UK & Northern &  & Upwards & Segmentation \\ 
   27 & \citealp{guell_petrongolo_2007} & ES & Southern &  & Upwards & Segmentation \\ 
   28 & \citealp{hagen_2002} & DE & Continental &  & Downwards & Integration \\ 
   29 & \citealp{hogberg_etal_2019} & 18 European countries & Northern &  & Upwards & Integration \\ 
   29 & \citealp{hogberg_etal_2019} & 18 European countries & Southern &  & Upwards & Segmentation \\ 
   29 & \citealp{hogberg_etal_2019} & 18 European countries & Continental &  & Upwards & Segmentation \\ 
   30 & \citealp{kiersztyn_2016} & PL & Continental &  & Upwards & Segmentation \\ 
   30 & \citealp{kiersztyn_2016} & PL & Continental &  & Upwards & Integration \\ 
   31 & \citealp{korpi_levin_2001} & SE & Northern &  & Downwards & null \\ 
   32 & \citealp{leschke_2009} & ES & Southern &  & None & Segmentation \\ 
   32 & \citealp{leschke_2009} & DE & Continental &  & None & Segmentation \\ 
   32 & \citealp{leschke_2009} & DK & Northern &  & None & Segmentation \\ 
   32 & \citealp{leschke_2009} & UK & Northern &  & None & Segmentation \\ 
   33 & \citealp{mcginnity_etal_2005} & DE-W & Continental &  & Upwards & Integration \\ 
   34 & \citealp{mertens_mcginnity_2002} & DE & Continental &  & Upwards & Integration \\ 
   35 & \citealp{mooi-reci_dekker_2015} & NL & Continental &  & None & Segmentation \\ 
   36 & \citealp{muffels_luijkx_2008} & 14 European countries & Northern &  & Upwards & Integration \\ 
   36 & \citealp{muffels_luijkx_2008} & 14 European countries & Southern &  & Upwards & Segmentation \\ 
   37 & \citealp{passaretta_wolbers_2019} & 17 European countries & Northern & small gap in EPL & Upwards & Integration \\ 
   37 & \citealp{passaretta_wolbers_2019} & 17 European countries & Southern & large gap in EPL & Upwards & Segmentation \\ 
   38 & \citealp{pavlopoulos_2013} & UK & Northern &  & Upwards & Segmentation \\ 
   38 & \citealp{pavlopoulos_2013} & DE & Continental &  & Upwards & Segmentation \\ 
   39 & \citealp{pfeifer_2012} & DE & Continental &  & Upwards & Segmentation \\ 
   40 & \citealp{picchio_2008} & IT & Southern &  & Upwards & Integration \\ 
   41 & \citealp{reichelt_2015} & DE & Continental & high- and low-skill & None & Segmentation \\ 
   41 & \citealp{reichelt_2015} & DE & Continental & medium skill & None & Integration \\ 
   42 & \citealp{remery_etal_2002} & NL & Continental &  & Upwards & Integration \\ 
   43 & \citealp{scherer_2004} & IT & Southern &  & Upwards & Segmentation \\ 
   43 & \citealp{scherer_2004} & DE-W & Continental &  & Upwards & Integration \\ 
   43 & \citealp{scherer_2004} & UK & Northern &  & Upwards & Segmentation \\ 
  \hline
\end{longtable}

}

\begin{table}[!h]
\rowcolors{8}{}{lightgray}
\caption{Empirical findings from the literature on the consequences of temporary employment using cross-national, comparative data}
    \resizebox{\textwidth}{!}{\begin{tabular}{lllll>{\raggedright\arraybackslash}p{1.2in}>{\raggedright\arraybackslash}p{1.2in}}
   \\[-1.8ex]\hline\hline \\ 
 [-1.8ex] \multicolumn{5}{l}{Article characteristics} 
& \multicolumn{2}{l}{Evidence (ref: Northern countries)}
\\ 

            \cmidrule(lr){1-5} 
            \cmidrule(lr){6-7} ID 
& Article 
& Countries
& Source
& Indicator 
& \multicolumn{1}{>{\raggedright\arraybackslash}p{1.2in}}{Continental disadvantage}
& \multicolumn{1}{>{\raggedright\arraybackslash}p{1.2in}}{Southern disadvantage}
             \\ 
 \hline
1 & \citealp{debels_2008} & EU & table 3.1 & men &  & yes \\ 
  1 & \citealp{debels_2008} & EU & table 3.1 & women & null & null \\ 
  2 & \citealp{gash_mcginnity_2007} & DE, FR &  & see note below &  &  \\ 
  3 & \citealp{gash_2008} & DK, FR, DE-W, UK & table 3 & permanent contract & null &  \\ 
  4 & \citealp{gebel_2010} & DE, UK & table 2 & wages & yes &  \\ 
  4 & \citealp{gebel_2010} & DE, UK & table 2 & permanent contract & yes &  \\ 
  5 & \citealp{giesecke_gross_2004} & DE, UK & text & wages & null &  \\ 
  5 & \citealp{giesecke_gross_2004} & DE, UK & text & permanent contract & null &  \\ 
  6 & \citealp{leschke_2009} & DE, DK, UK, ES & table 4 & permanent contract & mixed & yes \\ 
  7 & \citealp{muffels_luijkx_2008} & 14 European countries & table 3 & permanent contract & yes & null \\ 
  8 & \citealp{passaretta_wolbers_2019} & 17 European countries & see note below & permanent contract & yes & yes \\ 
  9 & \citealp{scherer_2004} & DE-W, UK, IT & abstract & wages & yes & yes \\ 
  9 & \citealp{scherer_2004} & DE-W, UK, IT & abstract & permanent contract & yes & yes \\ 
  10 & \citealp{babos_2014} & 8 CEE countries &  & see note below &  &  \\ 
  11 & \citealp{comi_grasseni_2012} & 9 European countries & text & wages & null & null \\ 
  12 & \citealp{pavlopoulos_2013} & DE, UK & text & wages & no &  \\ 
  13 & \citealp{gebel_2013} & CH, DE-W, UK & table 2 & wages & null &  \\ 
  13 & \citealp{gebel_2013} & CH, DE-W, UK & table 1 & permanent contract & null &  \\ 
  14 & \citealp{hogberg_etal_2019} & 18 European countries & figure 1 + text & permanent contract & yes & yes \\ 
  15 & \citealp{barbieri_cutuli_2016} & 13 European countries & table 1 + text & permanent contract & yes & yes \\ 
   \hline 
\end{tabular}
}
    \label{table_articles_comparative}
    \\
    \tiny
Note: No information exists for \citealp{babos_2014} because the article compares 8 countries, all of which are Central and Eastern European.  Similarly, no information exists for \citealp{gash_mcginnity_2007} because France is compared to Germany, both of which are Continental countries.  Information does exit for \citealp{passaretta_wolbers_2019} because we assume that Southern and Continental countries are more closed labor markets, with higher levels of segmentation between the primary and secondary labor market.  Therefore, the indication that gap in EPL matters can be understood as a disadvantage for Continental and Southern countries, even if the paper does not indicate a Southern or Continental disadvantage.
\end{table}


\begin{table}[!h]
\rowcolors{5}{}{lightgray}
\caption{Empirical findings from the literature on the consequences of temporary employment on wages}
    \resizebox{\textwidth}{!}{\begin{tabular}{l>{\raggedright\arraybackslash}p{2in}>{\raggedright\arraybackslash}p{1.25in}>{\raggedright\arraybackslash}p{.75in}l>{\raggedright\arraybackslash}p{1in}>{\raggedright\arraybackslash}p{1in}>{\raggedright\arraybackslash}p{1in}>{\raggedright\arraybackslash}p{1in}>{\raggedright\arraybackslash}p{1.3in}}
   \\[-1.8ex]\hline\hline \\ 
 [-1.8ex] \multicolumn{3}{l}{Article characteristics} 
& \multicolumn{3}{l}{Indicator characteristics} 
& \multicolumn{4}{l}{Evidence} 
\\ 

            \cmidrule(lr){1-3} 
            \cmidrule(lr){4-6}
            \cmidrule(lr){7-10} ID 
& Article 
& Source
& Country
& Indicator 
& Disadvantage 
& Men disadvantage
& Younger advantage
& High edu advantage
& \multicolumn{1}{>{\raggedright\arraybackslash}p{1.3in}}{Disadvantage declines over time}
             \\ 
 \hline
1 & \citealp{amuedo_dorantes_serrano_padial_2007} & table 5 + text (pg. 842) & ES & job movers & yes &  &  &  & no \\ 
  1 & \citealp{amuedo_dorantes_serrano_padial_2007} & table 5 + text (pg. 842) & ES & job stayers & yes &  &  &  & yes \\ 
  2 & \citealp{barbieri_cutuli_2018} & figure 5 + text & IT &  & yes &  &  &  &  \\ 
  3 & \citealp{berson_2018} & table 5 & FR &  & yes &  &  &  &  \\ 
  4 & \citealp{booth_etal_2002} & table 4 & UK & men & yes & no & no &  & yes \\ 
  4 & \citealp{booth_etal_2002} & table 4 & UK & women & yes &  & no &  & yes \\ 
  5 & \citealp{brown_sessions_2003} & table 4 & UK &  & yes &  &  & yes &  \\ 
  6 & \citealp{comi_grasseni_2012} & text & 9 European countries &  & yes &  &  &  &  \\ 
  7 & \citealp{de_graaf_zijl_etal_2011} & table 7 + text & NL &  & no &  &  &  &  \\ 
  8 & \citealp{de_lange_etal_2014} & table 4 & NL &  & null & no &  & yes & yes \\ 
  9 & \citealp{gash_mcginnity_2007} & table 4 & DE & women & no &  &  &  & no \\ 
  9 & \citealp{gash_mcginnity_2007} & table 5 & FR & women & null &  &  &  & null \\ 
  9 & \citealp{gash_mcginnity_2007} & table 5 & FR & men & null & null &  &  & null \\ 
  9 & \citealp{gash_mcginnity_2007} & table 4 & DE & men & yes & yes &  &  & yes \\ 
  10 & \citealp{gebel_2009} & table 5,6 + text & DE-W &  & yes &  &  & no &  \\ 
  11 & \citealp{gebel_2010} & table 2 & UK & men & null &  &  &  & null \\ 
  11 & \citealp{gebel_2010} & table 2 & DE & all & yes & yes &  & no & yes \\ 
  11 & \citealp{gebel_2010} & table 2 & DE & women & yes &  &  &  & yes \\ 
  11 & \citealp{gebel_2010} & table 2 & DE & men & yes &  &  &  & yes \\ 
  11 & \citealp{gebel_2010} & table 2 & UK & all & yes & no &  & no & yes \\ 
  11 & \citealp{gebel_2010} & table 2 & UK & women & yes &  &  &  & yes \\ 
  12 & \citealp{gebel_2013} & table 2 + text & CH &  & null &  &  &  & yes \\ 
  12 & \citealp{gebel_2013} & table 2 + text & DE-E &  & no &  &  &  & yes \\ 
  12 & \citealp{gebel_2013} & table 2 + text & DE-W &  & no &  &  &  & yes \\ 
  12 & \citealp{gebel_2013} & table 2 + text & UK &  & no &  &  &  & yes \\ 
  13 & \citealp{giesecke_gross_2004} & table 1 & DE & men & yes & null &  &  &  \\ 
  13 & \citealp{giesecke_gross_2004} & table 1 & DE & women & yes &  &  &  &  \\ 
  13 & \citealp{giesecke_gross_2004} & table 2 & UK & men & yes & yes &  &  &  \\ 
  13 & \citealp{giesecke_gross_2004} & table 2 & UK & women & no &  &  &  &  \\ 
  14 & \citealp{hagen_2002} & table 4 & DE &  & yes &  &  &  &  \\ 
  15 & \citealp{mertens_mcginnity_2002} & table 2b & DE-W & men & null &  &  &  &  \\ 
  15 & \citealp{mertens_mcginnity_2002} & table 2b & DE-W & women & null &  &  &  &  \\ 
  15 & \citealp{mertens_mcginnity_2002} & table 2b & DE-E & men & null &  &  &  &  \\ 
  15 & \citealp{mertens_mcginnity_2002} & table 2b & DE-E & women & null &  &  &  &  \\ 
  16 & \citealp{pavlopoulos_2013} & table 4 + text & UK & men & null & no & null & yes & null \\ 
  16 & \citealp{pavlopoulos_2013} & table 4 + text & DE & men & yes & yes & no & yes & yes \\ 
  16 & \citealp{pavlopoulos_2013} & table 4 + text & DE & women & yes &  & no & yes & yes \\ 
  16 & \citealp{pavlopoulos_2013} & table 4 + text & UK & women & yes &  & no & yes & yes \\ 
  17 & \citealp{pfeifer_2012} & table 2 & DE &  & yes & no & no & yes &  \\ 
  18 & \citealp{remery_etal_2002} & text (pg. 491) & NL &  & yes & yes &  &  &  \\ 
   \hline 
\end{tabular}
}
    \label{table_articles_wages}
    \\
\end{table}

\begin{table}[!h]
\rowcolors{5}{}{lightgray}
\caption{Empirical findings from the literature on the consequences of temporary employment on the transition into a permanent contract}
    \resizebox{\textwidth}{!}{\begin{tabular}{llll>{\raggedright\arraybackslash}p{1in}>{\raggedright\arraybackslash}p{1in}>{\raggedright\arraybackslash}p{1in}>{\raggedright\arraybackslash}p{1in}>{\raggedright\arraybackslash}p{1in}>{\raggedright\arraybackslash}p{1.3in}}
   \\[-1.8ex]\hline\hline \\ 
 [-1.8ex] \multicolumn{3}{l}{Article characteristics} 
& \multicolumn{3}{l}{Indicator characteristics} 
& \multicolumn{4}{l}{Evidence} 
\\ 

            \cmidrule(lr){1-3} 
            \cmidrule(lr){4-6}
            \cmidrule(lr){7-10} ID 
& Article 
& Source
& Country
& Indicator
& Disadvantage
& Men advantage
& Younger advantage
& High edu advantage
& \multicolumn{1}{>{\raggedright\arraybackslash}p{1.3in}}{Disadvantage declines over time}
             \\ 
 \hline
1 & \citealp{amuedo_dorantes_2000} & table 6 + text & ES &  & yes & yes & yes & null &  \\ 
  2 & \citealp{arranz_etal_2010} & table 3 & ES & men & yes & null & yes & null &  \\ 
  2 & \citealp{arranz_etal_2010} & table 4 & ES & women & null &  & null & null &  \\ 
  3 & \citealp{babos_2014} & table A.1 + text & 8 CEE countries &  & yes & null & no & null &  \\ 
  4 & \citealp{baranowska_etal_2011} & table 2 & PL &  &  & null &  & null &  \\ 
  5 & \citealp{barbieri_cutuli_2016} & table 1 + text & Southern &  & yes &  &  &  &  \\ 
  5 & \citealp{barbieri_cutuli_2016} & table 1 + text & Continental &  & yes &  &  &  &  \\ 
  5 & \citealp{barbieri_cutuli_2016} & table 1 + text & Northern &  & no &  &  &  &  \\ 
  6 & \citealp{barbieri_cutuli_2018} & figure 7 + text & IT &  & yes &  &  &  &  \\ 
  7 & \citealp{barbieri_scherer_2009} & table 2 & IT &  &  & yes & null & null &  \\ 
  8 & \citealp{barbieri_sestito_2008} & table 5 & IT &  &  & null & null &  &  \\ 
  9 & \citealp{berson_2018} & table 10 & FR &  & yes &  & no & no &  \\ 
  10 & \citealp{berton_etal_2011} & text & IT & Type of FTC: freelance contracts & yes &  &  &  &  \\ 
  10 & \citealp{berton_etal_2011} & text & IT & Type of FTC: fixed-term jobs, apprenticeship, and training programmes & no &  &  &  &  \\ 
  11 & \citealp{booth_etal_2002} & table 5 & UK & women &  &  & no & yes &  \\ 
  11 & \citealp{booth_etal_2002} & table 5 & UK & men &  &  & yes & null &  \\ 
  12 & \citealp{bosco_valeriani_2018} & table 5 + text & IT &  & yes & no & no & null &  \\ 
  13 & \citealp{de_graaf_zijl_etal_2011} & table 5 & NL &  &  & null & yes & yes &  \\ 
  14 & \citealp{de_lange_etal_2014} & table 2a & NL &  &  & null & null & null &  \\ 
  15 & \citealp{debels_2008} & table 3.1 & EU & men &  &  & null & yes &  \\ 
  15 & \citealp{debels_2008} & table 3.1 & EU & women &  &  & null & null &  \\ 
  16 & \citealp{gagliarducci_2005} & table 6 & IT &  &  & yes & no & null &  \\ 
  17 & \citealp{gash_2008} & table 2 & UK &  &  & null & no & yes &  \\ 
  17 & \citealp{gash_2008} & table 2 & DE-W &  &  & null & null & yes &  \\ 
  17 & \citealp{gash_2008} & table 2 & DK &  &  & null & null & null &  \\ 
  17 & \citealp{gash_2008} & table 2 & FR &  &  & null & null & null &  \\ 
  18 & \citealp{gash_mcginnity_2007} & table 7 + text & FR & men & no &  &  &  & no \\ 
  18 & \citealp{gash_mcginnity_2007} & table 6 + text & DE & women & null &  &  &  & null \\ 
  18 & \citealp{gash_mcginnity_2007} & table 6 + text & DE & men & yes &  &  &  & null \\ 
  18 & \citealp{gash_mcginnity_2007} & table 7 + text & FR & women & yes &  &  &  & yes \\ 
  19 & \citealp{gebel_2010} & table 2 & UK & men & yes &  &  &  & yes \\ 
  19 & \citealp{gebel_2010} & table 2 & DE & women & yes &  &  &  & yes \\ 
  19 & \citealp{gebel_2010} & table 2 & DE & all & yes & null &  & yes & yes \\ 
  19 & \citealp{gebel_2010} & table 2 & DE & men & yes &  &  &  & yes \\ 
  19 & \citealp{gebel_2010} & table 2 & UK & women & yes &  &  &  & yes \\ 
  19 & \citealp{gebel_2010} & table 2 & UK & all & yes & null &  & yes & yes \\ 
  20 & \citealp{gebel_2013} & table 1 + text & CH &  & null &  &  &  & null \\ 
  20 & \citealp{gebel_2013} & table 1 + text & DE-W &  & yes &  &  &  & yes \\ 
  20 & \citealp{gebel_2013} & table 1 + text & UK &  & yes &  &  &  & yes \\ 
  21 & \citealp{guell_petrongolo_2007} & table 8 & ES & men &  &  & no & yes &  \\ 
  21 & \citealp{guell_petrongolo_2007} & table 8 & ES & women &  &  & no & no &  \\ 
  21 & \citealp{guell_petrongolo_2007} & table 6 & ES & 1987-1998 & yes & null & no & null &  \\ 
  22 & \citealp{hogberg_etal_2019} & figure 1 + text & Southern &  & yes &  &  &  &  \\ 
  22 & \citealp{hogberg_etal_2019} & figure 1 + text & Continental &  & yes &  &  &  &  \\ 
  22 & \citealp{hogberg_etal_2019} & figure 1 + text & Northern &  & no &  &  &  &  \\ 
  23 & \citealp{kiersztyn_2016} & table 5 & PL & one-year &  & null & yes & yes &  \\ 
  23 & \citealp{kiersztyn_2016} & table 5 & PL & two-year &  & yes & null & yes &  \\ 
  24 & \citealp{mertens_mcginnity_2002} & table 6a & DE-W &  &  & null & no & no &  \\ 
  24 & \citealp{mertens_mcginnity_2002} & table 6b & DE-E &  &  & yes & null & null &  \\ 
  25 & \citealp{muffels_luijkx_2008} & table 3 & 14 EU countries & men &  &  & yes & no &  \\ 
  26 & \citealp{passaretta_wolbers_2019} & table 2 & 17 EU countries &  &  & yes & null & yes &  \\ 
  27 & \citealp{picchio_2008} & table 4 & IT &  &  & null & no & yes &  \\ 
  28 & \citealp{reichelt_2015} & table 2 & DE &  &  & no &  & no &  \\ 
  29 & \citealp{remery_etal_2002} & table 3 & NL &  & null & null & null & null &  \\ 
   \hline 
\end{tabular}
}
    \label{table_articles_permanent}
\end{table}

\begin{table}[!h]
\rowcolors{5}{}{lightgray}
\caption{Empirical findings from the literature on the consequences of temporary employment on the transition into unemployment}
    \resizebox{\textwidth}{!}{\begin{tabular}{llll>{\raggedright\arraybackslash}p{2in}l>{\raggedright\arraybackslash}p{1in}>{\raggedright\arraybackslash}p{1.5in}}
   \\[-1.8ex]\hline\hline \\ 
 [-1.8ex] \multicolumn{3}{l}{Article characteristics} 
& \multicolumn{2}{l}{Indicator characteristics} 
& \multicolumn{3}{l}{Evidence} 
\\ 

            \cmidrule(lr){1-3} 
            \cmidrule(lr){4-5}
            \cmidrule(lr){6-8} ID 
& Article 
& Source
& Country
& Gender 
& Disadvantage 
& \multicolumn{1}{>{\raggedright\arraybackslash}p{1in}}{Men advantage} 
& \multicolumn{1}{>{\raggedright\arraybackslash}p{1.5in}}{Disadvantage declines over time}
             \\ 
 \hline
1 & \citealp{amuedo_dorantes_2000} & table 6 + text & ES &  &  & no &  \\ 
  2 & \citealp{arranz_etal_2010} & table 3 + 4 & ES & age & yes &  &  \\ 
  3 & \citealp{barbieri_scherer_2009} & table 2 & IT &  &  & no &  \\ 
  4 & \citealp{berson_2018} & text & FR &  & yes &  &  \\ 
  5 & \citealp{de_lange_etal_2014} & table 2b & NE &  & yes & null & yes \\ 
  6 & \citealp{gash_mcginnity_2007} & table 4 & DE &  &  & null & yes \\ 
  6 & \citealp{gash_mcginnity_2007} & table 4 & DE &  &  & null & yes \\ 
  6 & \citealp{gash_mcginnity_2007} & table 4 & FR &  &  & null & yes \\ 
  6 & \citealp{gash_mcginnity_2007} & table 4 & FR &  &  & null & yes \\ 
  7 & \citealp{gebel_2010} & table 2 & DE & women & null & null & null \\ 
  7 & \citealp{gebel_2010} & table 2 & DE & men & null &  & null \\ 
  7 & \citealp{gebel_2010} & table 2 & UK & women & null &  & null \\ 
  7 & \citealp{gebel_2010} & table 2 & UK & men & yes & yes & no \\ 
  8 & \citealp{giesecke_gross_2003} &  & DE &  & yes &  &  \\ 
  9 & \citealp{giesecke_gross_2004} & table 5 & DE & men & yes & yes &  \\ 
  9 & \citealp{giesecke_gross_2004} & table 8 & UK & women & yes &  &  \\ 
  9 & \citealp{giesecke_gross_2004} & table 5 & DE & women & yes &  &  \\ 
  9 & \citealp{giesecke_gross_2004} & table 8 & UK & men & yes & yes &  \\ 
  10 & \citealp{hagen_2002} & text & DE &  & yes &  & yes \\ 
  11 & \citealp{hogberg_etal_2019} & text & 18 European countries & strict protection for regular contracts and partial deregulation & null &  &  \\ 
  12 & \citealp{mcginnity_etal_2005} & table 3 & DE-W &  & yes &  & yes \\ 
  13 & \citealp{mertens_mcginnity_2002} & table 6a & DE-W &  &  & null & yes \\ 
  13 & \citealp{mertens_mcginnity_2002} & table 6b & DE-E &  &  & null & yes \\ 
  14 & \citealp{mooi-reci_dekker_2015} & table 3a, 3b, and text & NE &  & yes &  &  \\ 
   \hline 
\end{tabular}
}
    \label{table_articles_unemployment}
\end{table}

\clearpage
\section{Appendix technical (online supplement)}
\label{appendix_technical}

Using all 43 articles that comprise our data, we conducted a text analysis where we do a word(s) search for 19 theories using 22 distinct word phrases.  The analysis was conducted using R.  We have included a do file for replication (text\_mining\_analysis.R), as well as the data (literature\_review\_corpus.rds).  For those who would like to replicate the data from scratch, load all 43 pdfs into a single folder and run the do file on the pdfs contained in that user specified folder.  The results from this analysis are contained in the file, article\_theories.xlsx, located in the online appendix.  

How does one know that it is the literature imposing the bridge-trap structure on us and not the other way around?  The results suggest that, in comparison to the theories we focus on in our review (`stepping stone', `trap', `segmentation', `integration', and `human capital', `signalling', and `screening'), all the other theories are used much less often.  Even where another theory is quite common, such as `job search,' the vast majority of uses are contained within two articles.  Furthermore, the theories we use cut across the discipline divide, and are well used across most articles in sociology, economics, and interdisciplinary journals.  The point of this analysis is to show that the theories we focus on are a reflection of the theories most commonly used in the literature to examine the consequences of temporary employment on wage and career mobility.

While we admit that a true text analysis would require its own article, which is beyond the scope of what we seek to do here, the results are suggestive of an important reality, one that provides the foundation for our analysis.  


